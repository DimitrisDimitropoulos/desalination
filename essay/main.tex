\documentclass[a4paper]{article}
\usepackage[fontsize=12pt]{fontsize}

%% Font and language settings
\usepackage{polyglossia}
\setdefaultlanguage{greek} \setotherlanguage{english}
\setmainfont{GFS Neohellenic} \setmonofont{Iosevka Nerd Font}
% \usepackage{cancel}
\usepackage{siunitx,physunits}
\usepackage{derivative}
\usepackage{unicode-math}
\setmathfont{GFS Neohellenic Math}
\usepackage{graphicx, caption}
% \usepackage{wrapfig, subcaption}
%% Table settings
\usepackage{array, booktabs, multirow, tabularx}
%% Hyperref settings
\usepackage{hyperref}
\hypersetup{
	colorlinks=true,
	linktoc=all,
	citecolor=blue,
	filecolor=blue,
	linkcolor=blue,
	urlcolor=blue
}
\hypersetup{linktocpage}
\def\numberanchor{section}
\numberwithin{equation}{\numberanchor}
\numberwithin{figure}{\numberanchor}
\numberwithin{table}{\numberanchor}

%% Typography settings
\usepackage[margin=1cm, includefoot, includehead]{geometry}
\usepackage[parfill]{parskip}
% \usepackage[protrusion=true]{microtype}

\usepackage{fancyhdr}
\pagestyle{fancy}
\rhead{Πρώτο Υποχρεωτικό Θέμα}

\usepackage{tikz}
\usepackage{pgfplots}
\usepackage{pgfplotstable}
\usepackage{pgf-pie}
\usepgfplotslibrary{dateplot}
\pgfplotsset{compat=1.18}



\begin{document}
\begin{titlepage}
	\begin{center}
		\begin{minipage}{0.15\textwidth}%
			\includegraphics[width=0.8\textwidth]{./logo/pyro.pdf}
		\end{minipage}\hspace{10pt}
		\begin{minipage}{0.6\textwidth}%
			ΕΘΝΙΚΟ ΜΕΤΣΟΒΙΟ ΠΟΛΥΤΕΧΝΕΙΟ\\
			ΣΧΟΛΗ ΜΗΧΑΝΟΛΟΓΩΝ ΜΗΧΑΝΙΚΩΝ\\
			ΤΟΜΕΑΣ ΡΕΥΣΤΩΝ\\
		\end{minipage}%
	\end{center}

	\par\noindent\rule{\textwidth}{0.8pt}

	\vspace{1cm}
	{\centering

		{\scshape\Large \underline{Πρώτο Υποχρεωτικό Θέμα}\par}
		\vspace{1.5cm}
		{\huge\bfseries Συνδυασμένη χρήση αιολικής και φωτοβολταϊκής ενέργειας για κάλυψη αναγκών μονάδας αφαλάτωσης\par}
		\vspace{2cm}
		{\Large\itshape~Δημήτριος Δημητρόπουλος mc21021\/\par}

		\vfill
		επιβλέπων\par
		Γιώργος Κάραλης

		\vfill

		\begin{center}
			{\large \today\\ Αθήνα}
		\end{center}
	}
\end{titlepage}



\section{Εισαγωγή}
Σε αυτή την εργασία εξετάζεται η περίπτωση χρήσης αφαλάτωσης συνδυαστικά με
ανανεώσιμες πηγές ενέργειας. Πιο συγκεκριμένα, μελετάτε η περίπτωση ενός
απομονωμένου δικτύου στηριζόμενο τόσο σε ανεμογεννήτριες όσο και σε
φωτοβολταϊκά με σκοπό την κάλυψη των αναγκών υδροδότησης ενός νησιού, χωρίς
κάποια αποθήκευση ενέργειας, με την χρήση μιας νέας δεξαμενής. Η μελέτη
στηρίζεται σε μία προσομοίωση λειτουργίας όλου του συστήματος για κάθε ώρα του
χρόνου, για διάφορα σενάρια. Εν συνεχεία, διαστασιολογείται το όλο σύστημα με
βάση ορισμένου οικονομοτεχνικούς δείκτες και όχι μόνο. Συνεπώς, προτείνεται ένα
σύστημα, το οποίο καλύπτει πλήρως τις ανάγκες του νησιού και σε βιώσιμο
οικονομικό επίπεδο για τα δεδομένα του νησιού.

\tableofcontents
\listoffigures
\listoftables

\clearpage

\section{Διαστασιολόγηση του συστήματος}
Στο εδάφιο αυτό παρουσιάζεται ενδελεχώς ή όλη διαδικασία επιλογής μεγεθών για
το προτεινόμενο σύστημα. Συγχρόνως, γίνεται αναφορά και στις υποθέσεις, που
έλαβαν χώρα προκειμένου να προχωρήσει η πρόταση.

\subsection{Υποθέσεις}
Για την εκτέλεση της προσομοίωσης του συστήματος έπρεπε να γίνουν ορισμένες
υποθέσεις, οι οποίες αφορούν περιφερειακές παραμέτρους. Καταρχάς, η αρχική
στάθμη της δεξαμενής θεωρείται ως \(10\%\) της τελικής στάθμης, για να μην
δημιουργηθούν προβλήματα κατά τις πρώτες ώρες της προσομοίωσης. Συγχρόνως, η
επιλογή της χαμηλής στάθμης έγινε με σκοπό να εξετασθεί μία δυσχερής περίπτωση.
Ειδικότερα, με μία μεγάλη γεμάτη δεξαμενή θα ήταν δυνατή μια φαινομενική κάλυψη
των αναγκών, η οποία όμως δεν ανταποκρίνεται στην πραγματικότητα, αφού το νερό
εμφανίζεται από το πουθενά. Τέλος, η χαμηλή αρχική στάθμη διασφαλίζει την
κάλυψη των αναγκών σε διαχρονικό επίπεδο και όχι μόνο διεποχικό.

Επιπλέον, το μοντέλο της ανεμογεννήτριας επιλέχθηκε από τις παρουσιάσεις του
μαθήματος ένα μοντέλο των \qty{1000}{\kilo\watt}. Η επιλογή αυτή βασίστηκε κατά
κύριο λόγο στην ισχύ της, η οποία προσφέρει ιδιαίτερη ευελιξία. Συγχρόνως, οι
υπολογισμοί βασίστηκαν στον πίνακα~\ref{tab:wind_turbine_data}, ο οποίος συνδέει την ταχύτητα του ανέμου με
την παραγόμενη ισχύ. Ειδικότερα, από αυτόν τον πίνακα κατασκευάστηκε μία cubic
spline, η οποία από την ταχύτητα του ανέμου δίνει την παραγόμενη ισχύ. Έτσι,
για κάθε σημείο της δεδομένης χρονοσειράς μπορεί να ληφθεί η ισχύ με μεγάλη
ακρίβεια, βλέπε~\ref{fig:wt_power_curve}.

\begin{table}[ht]
	\centering
	\caption{Ισχύς της ανεμογεννήτριας}\label{tab:wind_turbine_data}
	\footnotesize
	\begin{tabular}[H]{@{}ccccccccccccccccccccccccccc@{}}
		\toprule
		Ταχύτητα ανέμου \(\left[\si{\meter\per\second}\right]\) & 0 & 1 & 2 & 3 & 4  & 5  & 6   & 7   & 8   & 9   & 10  & 11  & 12  & 13  & 14  & 15--25 \\ \midrule \midrule
		Ισχύς \(\left[\si{\kilo\watt}\right]\)                  & 0 & 0 & 0 & 4 & 27 & 66 & 120 & 197 & 295 & 421 & 575 & 736 & 866 & 943 & 987 & 1000   \\ \bottomrule
	\end{tabular}
\end{table}

\begin{figure}[ht]
	\centering
	\begin{tikzpicture}
		\begin{axis}[
			width=0.7\textwidth,
			grid=major,
			xlabel={Ταχύτητα ανέμου [\(\si{\meter\per\second}\)]},
			ylabel={Ισχύς ανεμογεννήτριας [\(\si{\kilo\watt}\)]},
			title={Καμπύλη Ισχύος Ανεμογεννήτριας},
			unbounded coords=jump, % This option will make the top sides open
			]
			\addplot[
				only marks, % This option will plot only the points without connecting lines
				mark=*,
				mark options={scale=0.7}
			] table [x expr=\thisrow{v_wind}, y expr=\thisrow{power_wt}/21, col sep=comma] {../out/df_results.csv};
		\end{axis}
	\end{tikzpicture}
	\caption{Ισχύς της ανεμογεννήτριας για κάθε ώρα του χρόνου}\label{fig:wt_power_curve}
\end{figure}

Τέλος, για την μέγιστη ισχύ των φωτοβολταϊκών επιλέχθηκε η συνολική ισχύς των
φωτοβολταϊκών και όχι ανά πάνελ. Η επιλογή αυτή έγινε με σκοπό την απλοποίηση
των υπολογισμών, αφού η συνολική ισχύς των φωτοβολταϊκών είναι η ίδια
ανεξάρτητα από τον αριθμό των πάνελ. Επιπλέον, η επιλογή αυτή επιτρέπει την
εύκολη επέκταση του συστήματος, χωρίς να απαιτείται η αναθεώρηση των
υπολογισμών.



\section{Διαδικασία Υπολογισμών}

\subsection{Υπολογισμός Χωρητικότητας Δεξαμενής}
Για αρχή, ξεκινάμε με τον υπολογισμό της χωρητικότητας της δεξαμενής \(C_{\text{tank}}\).
Ειδικότερα, ο υπολογισμός αυτός βασίζεται στον περιορισμό πως πρέπει να δύναται
να παρέχει νερό για μία εβδομάδα του πιο απαιτητού μήνα χωρίς κάποια είσοδο.
Βέβαια, αυτό που γνωρίζουμε είναι η συνολική κατανάλωση και η μηνιαία
ποσοστιαία κατανομή της, το οποίο δημιουργεί πρόβλημα, αφού δεν έχουν όλοι οι
μήνες το ίδιο πλήθος ημερών. Επομένως, εξάγουμε την ημερήσια κατανάλωση και
συνεπώς την εβδομαδιαία για τον δυσκολότερο μήνα, η οποία αποτελεί το κάτω όριο
χωρητικότητας της δεξαμενής.

Η μηνιαία κατανάλωση νερού \(Q_{\text{monthly}, i}\) για κάθε μήνα \(i\), εκτιμάται ως εξής:

\begin{equation}\label{eq:monthly_water_consumption}
	Q_{\text{monthly}, i} = \frac{p_{\text{cons}, i}}{100} \cdot Q_{\text{yearly}}
\end{equation}

Όπου:
\begin{itemize}
	\item \(p_{\text{cons}, i}\) είναι το ποσοστό της συνολικής κατανάλωσης νερού για τον μήνα \(i\)
	\item \(Q_{\text{yearly}}\) είναι ο ετήσιος όγκος νερού που χρειάζεται σε εκατομμύρια κυβικά μέτρα
\end{itemize}

Ακόμη, η ημερήσια κατανάλωση νερού \(Q_{\text{daily}, i}\) για κάθε μήνα \(i\), έχει:

\begin{equation}\label{eq:daily_water_consumption}
	Q_{\text{daily}, i} = \frac{Q_{\text{monthly}, i}}{d_i}
\end{equation}

Όπου \(d_i\) είναι ο αριθμός των ημερών του μήνα \(i\). Προφανώς ο υπολογισμός
του πλέον απαιτητικού μήνα \(i_{\text{max}}\) είναι τετριμμένος.

Η εβδομαδιαία κατανάλωση νερού \(Q_{\text{weekly}}\) για τον πιο απαιτητικό μήνα έχει:
\begin{equation}\label{eq:weekly_water_consumption}
	Q_{\text{weekly}} = Q_{\text{daily}, i_{\text{max}}} \cdot 7
\end{equation}

Τέλος, προσθέτουμε κάποια επιπλέον χωρητικότητα για ασφάλεια και για οικονομία
όπως αποδεικνύεται στο τέλος. Συνεπώς:
\begin{equation}\label{eq:tank_capacity}
	C_{\text{tank}} = Q_{\text{weekly}} + Q_{\text{extra}}
\end{equation}

\subsection{Μηχανή Προσομοίωσης}
Σε αυτό το σημείο περιγράφουμε την καρδία των υπολογισμών, η οποία δεν είναι
άλλη από την διαδικασία προσομοίωσης (simulation engine). Αναλυτικότερα, η
μηχανή αυτή παρακολουθεί την στάθμη της δεξαμενής για κάθε ώρα του χρόνου,
έχοντας ως βάση τον ωριαίο ενεργειακό ισολογισμό στο σύνολο της εγκατάστασης.

Για αρχή, εκτιμούμε την εισερχόμενη στο σύστημα ενέργεια από τις χρονοσειρές,
που δίνονται. Πιο συγκεκριμένα, εκτιμούμε την ισχύ της ανεμογεννήτριας από την
spline, την οποία έχουμε κατασκευάσει προηγουμένως, και την ταχύτητα του ανέμου
για εκείνη την ώρα. Συγχρόνως, εκτιμούμε την ισχύ των φωτοβολταϊκών
πολλαπλασιάζοντας τις μετρήσεις με την μέγιστη εγκατεστημένη ισχύ, βλέπε~\eqref{eq:hourly_enegry_sum}.

\begin{equation}\label{eq:hourly_enegry_sum}
	E_{\text{in},h} = E_{\text{wind},h} + E_{\text{pv},h}= P_{\text{wt}} \left(v_{\text{wind},h}\right) + \frac{p_{\text{pv},h}}{1000}\cdot P_{\text{pv,peak}}\quad \left[\si{\kilo\watt}\right]
\end{equation}

Στην συνέχεια, υπολογίζουμε πόσοι σταθμοί αφαλάτωσης είναι ενεργοί για κάθε
ώρα. Αναλυτικότερα, ξεκινάμε με τον υπολογισμό της ωριαίας κατανάλωσης μίας
μονάδας αφαλάτωσης, ως:

\begin{equation}\label{eq:hourly_desalination_consumption}
	E_{\rho ,h}= \frac{q_{\rho}}{24}\cdot \sigma_{\rho}\quad \left[\si{\kilo\watt}\right]
\end{equation}

Όπου \(q_{\rho}\) είναι η ημερήσια δυναμικότητα της μονάδας αφαλάτωσης και
\(\sigma_{\rho}\) η ειδική κατανάλωση της. Επομένως, συγκρίνοντας την
προσφερόμενη με την καταναλισκόμενη ισχύ είμαστε σε θέση να υπολογίσουμε τους
σταθμούς που λειτουργούν ανά ώρα. Μετά, είναι εφικτός ο υπολογισμός και του
παραγόμενου νερού ανά ώρα:

\begin{equation}\label{eq:hourly_desalination_production}
	Q_{\rho , h}=n_{\rho ,h}\cdot \frac{q_{\rho}}{24}\quad \left[\si{\cubic\meter\per\hour}\right]
\end{equation}

Τέλος, είμαστε σε θέση να υπολογίσουμε την στάθμη της δεξαμενής για κάθε ώρα~\ref{fig:tank_level}.
Πιο ειδικά, ο υπολογισμός της στάθμης στηρίζεται στον ωριαίο ισολογισμό μάζας
μεταξύ ζήτησης και παραγωγής.

\begin{figure}
	\centering
	\begin{tikzpicture}
		\begin{axis}[
				width=0.7\textwidth,
				grid=major,
				xlabel={Ημερομηνία και Ώρα},
				ylabel={Όγκος στην δεξαμενή \(\left[\si{\cubic\meter}\right]\)},
				title={Επίπεδο Δεξαμενής με την Πάροδο του Χρόνου},
				date coordinates in=x,
				table/col sep=comma,
				date ZERO=2023-01-01,
				xticklabel=\month-\day,
				xticklabel style={rotate=45, anchor=near xticklabel},
				unbounded coords=jump,
				xmin=2023-01-01,
				xmax=2023-12-31,
			]
			\addplot+[
				no markers,
				black,
				thick
			] table[x=datetime,y=tank_level] {../out/df_results.csv};
		\end{axis}
	\end{tikzpicture}
	\caption{Όγκος νερού στη δεξαμενή ανά ώρα του χρόνου}\label{fig:tank_level}
\end{figure}

\subsection{Επιλογή Μεγεθών Συστήματος}

Σε αυτό το σημείο, είμαστε σε θέση να επιλέξουμε τα μεγέθη του συστήματος,
εφορμόμενοι από την μηχανή προσομοίωσης. Ειδικότερα, τρέχουμε πολλές φορές την
μηχανή για ένα διακριτό εύρος των παραμέτρων της συνολικής ισχύς των
φωτοβολταϊκών και του αριθμού των ανεμογεννητριών που θα βάλουμε. Την
διαδικασία αυτή την κάνουμε τρεις φορές για τις περιπτώσεις του πλήθους των
σταθμών αφαλάτωσης. Συγχρόνως, θέτουμε και κριτήριο βιωσιμότητας της
εγκατάστασης να μην πέσει η στάθμη κάτω από \qty{1000}{\cubic\meter}. Τέλος,
υπολογίζουμε και το CAPEX, όπως θα αναφερθεί και παρακάτω. Συμπερασματικά,
έχουμε το CAPEX για όλα τα κύρια σενάρια, πλήθους ανεμογεννητριών,
φωτοβολταϊκών και σταθμών αφαλάτωσης, κάτι το οποίο μας δίνει τη δυνατότητα
σύγκρισης και επιλογής με βάση το χαμηλότερο CAPEX~\ref{fig:viable_solutions_3}
και~\ref{fig:viable_solutions_4}.

Αξίζει να σημειωθεί, ότι η διαδικασία αυτή είναι πολύπλοκη και απαιτεί
σημαντικό χρόνο εκτέλεσης. Για τον λόγο αυτό, οι υπολογισμοί αυτοί υλοποιήθηκαν
σε παράλληλη επεξεργασία. Πιο συγκεκριμένα, κάθε νήμα τρέχει μία προσομοίωση
για κάποιες συνθήκες, οι οποίες συγκρίνονται, βάσει CAPEX στο τέλος.

\begin{figure}
	\centering
	\begin{tikzpicture}
		\begin{axis}[
				width=0.7\textwidth,
				view={0}{90},
				grid=major,
				colorbar,
				colormap/viridis,
				xlabel={\(n_{wt}\)},
				ylabel={\(p_{pv} \left[\si{\kilo\watt}\right]\)},
				zlabel={CAPEX},
			]
			\addplot3[
				scatter,
				only marks,
				scatter src=explicit,
				mark size=4,
			]
			table[
					x=n_wind,
					y=p_solar,
					meta=total_capex,
					col sep=comma,
				] {../out/viable_solutions.csv};
		\end{axis}
	\end{tikzpicture}
	\caption{Σύγκριση των λύσεων με βάση το CAPEX για 3 σταθμούς αφαλάτωσης, χαμηλότερο είναι καλύτερο}\label{fig:viable_solutions_3}
\end{figure}

\begin{figure}
	\centering
	\begin{tikzpicture}
		\begin{axis}[
				width=0.7\textwidth,
				view={0}{90},
				grid=major,
				colorbar,
				colormap/viridis,
				xlabel={\(n_{wt}\)},
				ylabel={\(p_{pv} \left[\si{\kilo\watt}\right]\)},
				zlabel={CAPEX},
			]
			\addplot3[
				scatter,
				only marks,
				scatter src=explicit,
				mark size=4,
			]
			table[
					x=n_wind,
					y=p_solar,
					meta=total_capex,
					col sep=comma,
				] {../out/viable_solutions_4.csv};
		\end{axis}
	\end{tikzpicture}
	\caption{Σύγκριση των λύσεων με βάση το CAPEX για 4 σταθμούς αφαλάτωσης, χαμηλότερο είναι καλύτερο}\label{fig:viable_solutions_4}
\end{figure}


\section{Κόστος Παραγόμενου Νερού}

Σε αυτό το σημείο γίνεται η ανάλυση του κάθε κόστους με σκοπό την εύρεση του
κόστους του νερού που παράγεται ανά κυβικό.

\section{Αρχικό Κόστος Επένδυσης}

Για αρχή, θα υπολογίσουμε το κόστος της δεξαμενής, με βάση την χωρητικότητα της:

\begin{equation}\label{eq:tank_cost}
	C_R = 1000 \cdot V^{0.6}
\end{equation}

Στη συνέχεια, βρίσκουμε το κόστος των φωτοβολταϊκών, από την μέγιστη εγκατεστημένη ισχύ:

\begin{equation}\label{eq:pv_cost}
	C_{pv}= P_{\text{pv,peak}}\cdot c_{\text{pv}}
\end{equation}

Κατόπιν, υπολογίζουμε το κόστος των ανεμογεννητριών, από την εγκατεστημένη ισχύ:

\begin{equation}\label{eq:wt_cost}
	C_{wt}= P_{\text{wt}}\cdot c_{\text{wt}}
\end{equation}

Τέλος, υπολογίζουμε και το κόστος της αφαλάτωσης, από το ανηγμένο αρχικό κόστος
για μονάδα αφαλάτωσης:

\begin{equation}\label{eq:desalination_cost}
	C_{\rho}= c_{\rho} \cdot n_{\rho} \cdot q_{\rho}
\end{equation}

Όπου, \(c_{\rho}\) το ανηγμένο CAPEX, \(n_{\rho}\) ο αριθμός των μονάδων
αφαλάτωσης και \(q_{\rho}\) η ημερήσια δυναμικότητα

Τελικά, το συνολικό αρχικό κόστος επένδυσης, όπως φαίνεται και εδώ~\ref{fig:CAPEX} είναι:

\begin{equation}\label{eq:total_CAPEX}
	\text{CAPEX} = C_R + C_{pv} + C_{wt} + C_{\rho}
\end{equation}

\begin{figure}[ht]
	\centering
	\begin{tikzpicture}
		\pie[
			width=0.7\textwidth,
			text=legend,
			radius=4,
			color={blue!30, red!30, green!30, yellow!30},
			sum=auto,
			% title=Συνολικό Κόστος Επένδυσης \(\qty{103.6}{\million\euro}\)
		]{
			27.23/Ανεμογεννήτριες,
			18.96/Φωτοβολταϊκά,
			50.16/Aφαλάτωση,
			3.66/Δεξαμενή
		}
	\end{tikzpicture}
	\caption{Ποσοστιαία κατανομή του αρχικού κόστους επένδυσης}\label{fig:CAPEX}
\end{figure}

\subsection{Κόστος Παραγόμενου Νερού}

Το λειτουργικό κόστος της αφαλάτωσης, υπολογίζεται σαν ποσοστό του αρχικού:

\begin{equation}\label{eq:OPEX}
	\text{OPEX}=3\%\cdot \text{CAPEX}
\end{equation}

Τελικά, το κόστος παραγόμενου νερού, για εικοσαετή ορίζοντα \(Ν=20\) με
επιτόκιο προεξόφλησης \(i=6\%\):

\begin{equation}\label{eq:LCOW}
	\text{LCOW}= \frac{\text{CAPEX}\cdot R + \text{OPEX}}{W}, \quad R=\frac{i}{1-(1+i)^{-N}}
\end{equation}



\end{document}
