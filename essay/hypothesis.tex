\subsection{Υποθέσεις}
Για την εκτέλεση της προσομοίωσης του συστήματος έπρεπε να γίνουν ορισμένες
υποθέσεις, οι οποίες αφορούν περιφερειακές παραμέτρους. Καταρχάς, η αρχική
στάθμη της δεξαμενής θεωρείται ως \(10\%\) της τελικής στάθμης, για να μην
δημιουργηθούν προβλήματα κατά τις πρώτες ώρες της προσομοίωσης. Συγχρόνως, η
επιλογή της χαμηλής στάθμης έγινε με σκοπό να εξετασθεί μία δυσχερής περίπτωση.
Ειδικότερα, με μία μεγάλη γεμάτη δεξαμενή θα ήταν δυνατή μια φαινομενική κάλυψη
των αναγκών, η οποία όμως δεν ανταποκρίνεται στην πραγματικότητα, αφού το νερό
εμφανίζεται από το πουθενά. Τέλος, η χαμηλή αρχική στάθμη διασφαλίζει την
κάλυψη των αναγκών σε διαχρονικό επίπεδο και όχι μόνο διεποχικό.

Επιπλέον, το μοντέλο της ανεμογεννήτριας επιλέχθηκε από τις παρουσιάσεις του
μαθήματος ένα μοντέλο των \qty{1000}{\kilo\watt}. Η επιλογή αυτή βασίστηκε κατά
κύριο λόγο στην ισχύ της, η οποία προσφέρει ιδιαίτερη ευελιξία. Συγχρόνως, οι
υπολογισμοί βασίστηκαν στον πίνακα~\ref{tab:wind_turbine_data}, ο οποίος συνδέει την ταχύτητα του ανέμου με
την παραγόμενη ισχύ. Ειδικότερα, από αυτόν τον πίνακα κατασκευάστηκε μία cubic
spline, η οποία από την ταχύτητα του ανέμου δίνει την παραγόμενη ισχύ. Έτσι,
για κάθε σημείο της δεδομένης χρονοσειράς μπορεί να ληφθεί η ισχύ με μεγάλη
ακρίβεια, βλέπε~\ref{fig:wt_power_curve}.

\begin{table}[ht]
	\centering
	\caption{Ισχύς της ανεμογεννήτριας}\label{tab:wind_turbine_data}
	\footnotesize
	\begin{tabular}[H]{@{}ccccccccccccccccccccccccccc@{}}
		\toprule
		Ταχύτητα ανέμου \(\left[\si{\meter\per\second}\right]\) & 0 & 1 & 2 & 3 & 4  & 5  & 6   & 7   & 8   & 9   & 10  & 11  & 12  & 13  & 14  & 15--25 \\ \midrule \midrule
		Ισχύς \(\left[\si{\kilo\watt}\right]\)                  & 0 & 0 & 0 & 4 & 27 & 66 & 120 & 197 & 295 & 421 & 575 & 736 & 866 & 943 & 987 & 1000   \\ \bottomrule
	\end{tabular}
\end{table}

\begin{figure}[ht]
	\centering
	\begin{tikzpicture}
		\begin{axis}[
			width=0.7\textwidth,
			grid=major,
			xlabel={Ταχύτητα ανέμου [\(\si{\meter\per\second}\)]},
			ylabel={Ισχύς ανεμογεννήτριας [\(\si{\kilo\watt}\)]},
			title={Καμπύλη Ισχύος Ανεμογεννήτριας},
			unbounded coords=jump, % This option will make the top sides open
			]
			\addplot[
				only marks, % This option will plot only the points without connecting lines
				mark=*,
				mark options={scale=0.7}
			] table [x expr=\thisrow{v_wind}, y expr=\thisrow{power_wt}/21, col sep=comma] {../out/df_results.csv};
		\end{axis}
	\end{tikzpicture}
	\caption{Ισχύς της ανεμογεννήτριας για κάθε ώρα του χρόνου}\label{fig:wt_power_curve}
\end{figure}

Τέλος, για την μέγιστη ισχύ των φωτοβολταϊκών επιλέχθηκε η συνολική ισχύς των
φωτοβολταϊκών και όχι ανά πάνελ. Η επιλογή αυτή έγινε με σκοπό την απλοποίηση
των υπολογισμών, αφού η συνολική ισχύς των φωτοβολταϊκών είναι η ίδια
ανεξάρτητα από τον αριθμό των πάνελ. Επιπλέον, η επιλογή αυτή επιτρέπει την
εύκολη επέκταση του συστήματος, χωρίς να απαιτείται η αναθεώρηση των
υπολογισμών.

