\section{Διαδικασία Υπολογισμών}

\subsection{Υπολογισμός Χωρητικότητας Δεξαμενής}
Για αρχή, ξεκινάμε με τον υπολογισμό της χωρητικότητας της δεξαμενής \(C_{\text{tank}}\).
Ειδικότερα, ο υπολογισμός αυτός βασίζεται στον περιορισμό πως πρέπει να δύναται
να παρέχει νερό για μία εβδομάδα του πιο απαιτητού μήνα χωρίς κάποια είσοδο.
Βέβαια, αυτό που γνωρίζουμε είναι η συνολική κατανάλωση και η μηνιαία
ποσοστιαία κατανομή της, το οποίο δημιουργεί πρόβλημα, αφού δεν έχουν όλοι οι
μήνες το ίδιο πλήθος ημερών. Επομένως, εξάγουμε την ημερήσια κατανάλωση και
συνεπώς την εβδομαδιαία για τον δυσκολότερο μήνα, η οποία αποτελεί το κάτω όριο
χωρητικότητας της δεξαμενής.

Η μηνιαία κατανάλωση νερού \(Q_{\text{monthly}, i}\) για κάθε μήνα \(i\), εκτιμάται ως εξής:

\begin{equation}\label{eq:monthly_water_consumption}
	Q_{\text{monthly}, i} = \frac{p_{\text{cons}, i}}{100} \cdot Q_{\text{yearly}}
\end{equation}

Όπου:
\begin{itemize}
	\item \(p_{\text{cons}, i}\) είναι το ποσοστό της συνολικής κατανάλωσης νερού για τον μήνα \(i\)
	\item \(Q_{\text{yearly}}\) είναι ο ετήσιος όγκος νερού που χρειάζεται σε εκατομμύρια κυβικά μέτρα
\end{itemize}

Ακόμη, η ημερήσια κατανάλωση νερού \(Q_{\text{daily}, i}\) για κάθε μήνα \(i\), έχει:

\begin{equation}\label{eq:daily_water_consumption}
	Q_{\text{daily}, i} = \frac{Q_{\text{monthly}, i}}{d_i}
\end{equation}

Όπου \(d_i\) είναι ο αριθμός των ημερών του μήνα \(i\). Προφανώς ο υπολογισμός
του πλέον απαιτητικού μήνα \(i_{\text{max}}\) είναι τετριμμένος.

Η εβδομαδιαία κατανάλωση νερού \(Q_{\text{weekly}}\) για τον πιο απαιτητικό μήνα έχει:
\begin{equation}\label{eq:weekly_water_consumption}
	Q_{\text{weekly}} = Q_{\text{daily}, i_{\text{max}}} \cdot 7
\end{equation}

Τέλος, προσθέτουμε κάποια επιπλέον χωρητικότητα για ασφάλεια και για οικονομία
όπως αποδεικνύεται στο τέλος. Συνεπώς:
\begin{equation}\label{eq:tank_capacity}
	C_{\text{tank}} = Q_{\text{weekly}} + Q_{\text{extra}}
\end{equation}

\subsection{Μηχανή Προσομοίωσης}
Σε αυτό το σημείο περιγράφουμε την καρδία των υπολογισμών, η οποία δεν είναι
άλλη από την διαδικασία προσομοίωσης (simulation engine). Αναλυτικότερα, η
μηχανή αυτή παρακολουθεί την στάθμη της δεξαμενής για κάθε ώρα του χρόνου,
έχοντας ως βάση τον ωριαίο ενεργειακό ισολογισμό στο σύνολο της εγκατάστασης.

Για αρχή, εκτιμούμε την εισερχόμενη στο σύστημα ενέργεια από τις χρονοσειρές,
που δίνονται. Πιο συγκεκριμένα, εκτιμούμε την ισχύ της ανεμογεννήτριας από την
spline, την οποία έχουμε κατασκευάσει προηγουμένως, και την ταχύτητα του ανέμου
για εκείνη την ώρα. Συγχρόνως, εκτιμούμε την ισχύ των φωτοβολταϊκών
πολλαπλασιάζοντας τις μετρήσεις με την μέγιστη εγκατεστημένη ισχύ, βλέπε~\eqref{eq:hourly_enegry_sum}.

\begin{equation}\label{eq:hourly_enegry_sum}
	E_{\text{in},h} = E_{\text{wind},h} + E_{\text{pv},h}= P_{\text{wt}} \left(v_{\text{wind},h}\right) + \frac{p_{\text{pv},h}}{1000}\cdot P_{\text{pv,peak}}\quad \left[\si{\kilo\watt}\right]
\end{equation}

Στην συνέχεια, υπολογίζουμε πόσοι σταθμοί αφαλάτωσης είναι ενεργοί για κάθε
ώρα~\ref{fig:working_hours_per_des}. Σε αυτό το σημείο αξίζει να τονίσουμε πως
από το πλήθος των μονάδων, οι οποίες χρησιμοποιούνται σε ωριαία βάση είναι
εμφανές πως δεν σπαταλάτε η ενέργεια που παράγεται. Αναλυτικότερα, ξεκινάμε με
τον υπολογισμό της ωριαίας κατανάλωσης μίας μονάδας αφαλάτωσης, ως:

\begin{equation}\label{eq:hourly_desalination_consumption}
	E_{\rho ,h}= \frac{q_{\rho}}{24}\cdot \sigma_{\rho}\quad \left[\si{\kilo\watt}\right]
\end{equation}

\begin{figure}[ht]
	\centering
	\begin{tikzpicture}
		\begin{axis}[
				width=0.7\textwidth,
				ybar,
				symbolic x coords={0, 1, 2, 3},
				xtick=data,
				xlabel={Αριθμός Λειτουργούντων Σταθμών},
				ylabel={Ώρες},
				nodes near coords,
				ymin=0,
				bar width=20pt,
				enlarge x limits=0.25,
				title={Αριθμός Ωρών Λειτουργίας ανά Πλήθος Σταθμών},
			]
			\addplot coordinates {(0, 2354) (1, 1126) (2, 898) (3, 4382)};
		\end{axis}
	\end{tikzpicture}
	\caption{Αριθμός ωρών που λειτουργούν οι σταθμοί αφαλάτωσης ανά πλήθος σταθμών}\label{fig:working_hours_per_des}
\end{figure}

Όπου \(q_{\rho}\) είναι η ημερήσια δυναμικότητα της μονάδας αφαλάτωσης και
\(\sigma_{\rho}\) η ειδική κατανάλωση της. Επομένως, συγκρίνοντας την
προσφερόμενη με την καταναλισκόμενη ισχύ είμαστε σε θέση να υπολογίσουμε τους
σταθμούς που λειτουργούν ανά ώρα. Μετά, είναι εφικτός ο υπολογισμός και του
παραγόμενου νερού ανά ώρα:

\begin{equation}\label{eq:hourly_desalination_production}
	Q_{\rho , h}=n_{\rho ,h}\cdot \frac{q_{\rho}}{24}\quad \left[\si{\cubic\meter\per\hour}\right]
\end{equation}

Τέλος, είμαστε σε θέση να υπολογίσουμε την στάθμη της δεξαμενής για κάθε ώρα~\ref{fig:tank_level}.
Πιο ειδικά, ο υπολογισμός της στάθμης στηρίζεται στον ωριαίο ισολογισμό μάζας
μεταξύ ζήτησης και παραγωγής.

\begin{figure}[ht]
	\centering
	\begin{tikzpicture}
		\begin{axis}[
				width=0.7\textwidth,
				grid=major,
				xlabel={Ημερομηνία και Ώρα},
				ylabel={Όγκος στην δεξαμενή \(\left[\si{\cubic\meter}\right]\)},
				title={Επίπεδο Δεξαμενής με την Πάροδο του Χρόνου},
				date coordinates in=x,
				table/col sep=comma,
				date ZERO=2023-01-01,
				xticklabel=\month-\day,
				xticklabel style={rotate=45, anchor=near xticklabel},
				unbounded coords=jump,
				xmin=2023-01-01,
				xmax=2023-12-31,
			]
			\addplot+[
				no markers,
				black,
				thick
			] table[x=datetime,y=tank_level] {../out/df_results.csv};
		\end{axis}
	\end{tikzpicture}
	\caption{Όγκος νερού στη δεξαμενή ανά ώρα του χρόνου}\label{fig:tank_level}
\end{figure}

\subsection{Επιλογή Μεγεθών Συστήματος}

Σε αυτό το σημείο, είμαστε σε θέση να επιλέξουμε τα μεγέθη του συστήματος,
εφορμόμενοι από την μηχανή προσομοίωσης. Ειδικότερα, τρέχουμε πολλές φορές την
μηχανή για ένα διακριτό εύρος των παραμέτρων της συνολικής ισχύς των
φωτοβολταϊκών και του αριθμού των ανεμογεννητριών που θα βάλουμε. Την
διαδικασία αυτή την κάνουμε τρεις φορές για τις περιπτώσεις του πλήθους των
σταθμών αφαλάτωσης. Συγχρόνως, θέτουμε και κριτήριο βιωσιμότητας της
εγκατάστασης να μην πέσει η στάθμη κάτω από \qty{1000}{\cubic\meter}. Τέλος,
υπολογίζουμε και το CAPEX, όπως θα αναφερθεί και παρακάτω. Συμπερασματικά,
έχουμε το CAPEX για όλα τα κύρια σενάρια, πλήθους ανεμογεννητριών,
φωτοβολταϊκών και σταθμών αφαλάτωσης, κάτι το οποίο μας δίνει τη δυνατότητα
σύγκρισης και επιλογής με βάση το χαμηλότερο CAPEX~\ref{fig:viable_solutions_3}
και~\ref{fig:viable_solutions_4}.

Αξίζει να σημειωθεί, ότι η διαδικασία αυτή είναι πολύπλοκη και απαιτεί
σημαντικό χρόνο εκτέλεσης. Για τον λόγο αυτό, οι υπολογισμοί αυτοί υλοποιήθηκαν
σε παράλληλη επεξεργασία. Πιο συγκεκριμένα, κάθε νήμα τρέχει μία προσομοίωση
για κάποιες συνθήκες, οι οποίες συγκρίνονται, βάσει CAPEX στο τέλος.

\begin{figure}[ht]
	\centering
	\begin{tikzpicture}
		\begin{axis}[
				width=0.7\textwidth,
				view={0}{90},
				grid=major,
				colorbar,
				colormap/viridis,
				xlabel={\(n_{wt}\)},
				ylabel={\(p_{pv} \left[\si{\kilo\watt}\right]\)},
				zlabel={CAPEX},
			]
			\addplot3[
				scatter,
				only marks,
				scatter src=explicit,
				mark size=4,
			]
			table[
					x=n_wind,
					y=p_solar,
					meta=total_capex,
					col sep=comma,
				] {../out/viable_solutions.csv};
		\end{axis}
	\end{tikzpicture}
	\caption{Σύγκριση των λύσεων με βάση το CAPEX για 3 σταθμούς αφαλάτωσης, χαμηλότερο είναι καλύτερο}\label{fig:viable_solutions_3}
\end{figure}

\begin{figure}[ht]
	\centering
	\begin{tikzpicture}
		\begin{axis}[
				width=0.7\textwidth,
				view={0}{90},
				grid=major,
				colorbar,
				colormap/viridis,
				xlabel={\(n_{wt}\)},
				ylabel={\(p_{pv} \left[\si{\kilo\watt}\right]\)},
				zlabel={CAPEX},
			]
			\addplot3[
				scatter,
				only marks,
				scatter src=explicit,
				mark size=4,
			]
			table[
					x=n_wind,
					y=p_solar,
					meta=total_capex,
					col sep=comma,
				] {../out/viable_solutions_4.csv};
		\end{axis}
	\end{tikzpicture}
	\caption{Σύγκριση των λύσεων με βάση το CAPEX για 4 σταθμούς αφαλάτωσης, χαμηλότερο είναι καλύτερο}\label{fig:viable_solutions_4}
\end{figure}
