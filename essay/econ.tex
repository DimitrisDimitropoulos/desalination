\section{Κόστος Παραγόμενου Νερού}

Σε αυτό το σημείο γίνεται η ανάλυση του κάθε κόστους με σκοπό την εύρεση του
κόστους του νερού που παράγεται ανά κυβικό.

\section{Αρχικό Κόστος Επένδυσης}

Για αρχή, θα υπολογίσουμε το κόστος της δεξαμενής, με βάση την χωρητικότητα της:

\begin{equation}\label{eq:tank_cost}
	C_R = 1000 \cdot V^{0.6}
\end{equation}

Στη συνέχεια, βρίσκουμε το κόστος των φωτοβολταϊκών, από την μέγιστη εγκατεστημένη ισχύ:

\begin{equation}\label{eq:pv_cost}
	C_{pv}= P_{\text{pv,peak}}\cdot c_{\text{pv}}
\end{equation}

Κατόπιν, υπολογίζουμε το κόστος των ανεμογεννητριών, από την εγκατεστημένη ισχύ:

\begin{equation}\label{eq:wt_cost}
	C_{wt}= P_{\text{wt}}\cdot c_{\text{wt}}
\end{equation}

Τέλος, υπολογίζουμε και το κόστος της αφαλάτωσης, από το ανηγμένο αρχικό κόστος
για μονάδα αφαλάτωσης:

\begin{equation}\label{eq:desalination_cost}
	C_{\rho}= c_{\rho} \cdot n_{\rho} \cdot q_{\rho}
\end{equation}

Όπου, \(c_{\rho}\) το ανηγμένο CAPEX, \(n_{\rho}\) ο αριθμός των μονάδων
αφαλάτωσης και \(q_{\rho}\) η ημερήσια δυναμικότητα

Τελικά, το συνολικό αρχικό κόστος επένδυσης, όπως φαίνεται και εδώ~\ref{fig:CAPEX} είναι:

\begin{equation}\label{eq:total_CAPEX}
	\text{CAPEX} = C_R + C_{pv} + C_{wt} + C_{\rho}
\end{equation}

\begin{figure}[ht]
	\centering
	\begin{tikzpicture}
		\pie[
			width=0.7\textwidth,
			text=legend,
			radius=4,
			color={blue!30, red!30, green!30, yellow!30},
			sum=auto,
			% title=Συνολικό Κόστος Επένδυσης \(\qty{103.6}{\million\euro}\)
		]{
			27.23/Ανεμογεννήτριες,
			18.96/Φωτοβολταϊκά,
			50.16/Aφαλάτωση,
			3.66/Δεξαμενή
		}
	\end{tikzpicture}
	\caption{Ποσοστιαία κατανομή του αρχικού κόστους επένδυσης}\label{fig:CAPEX}
\end{figure}

\subsection{Κόστος Παραγόμενου Νερού}

Το λειτουργικό κόστος της αφαλάτωσης, υπολογίζεται σαν ποσοστό του αρχικού:

\begin{equation}\label{eq:OPEX}
	\text{OPEX}=3\%\cdot \text{CAPEX}
\end{equation}

Τελικά, το κόστος παραγόμενου νερού, για εικοσαετή ορίζοντα \(Ν=20\) με
επιτόκιο προεξόφλησης \(i=6\%\):

\begin{equation}\label{eq:LCOW}
	\text{LCOW}= \frac{\text{CAPEX}\cdot R + \text{OPEX}}{W}, \quad R=\frac{i}{1-(1+i)^{-N}}
\end{equation}

